%%%%%%%%%%%%%%%%%%%%%%            -*-LaTeX-*-
%%%   manual.tex   %%%
%%%%%%%%%%%%%%%%%%%%%%
%%
%%  Date:   06-Mar-2002
%%  Author: wd (Wolfgang.Dobler@kis.uni-freiburg.de)
%%  Description: 
% 
\ifx\pdfoutput\undefined        % not running pdflatex
  \documentclass[12pt,twoside,notitlepage,a4paper]{article}
\else                           % running pdflatex
  \documentclass[pdftex,12pt,twoside,notitlepage,a4paper]{article}
\fi

\usepackage[bookmarks=false]{hyperref}
%\usepackage{german,a4}
%\usepackage[german,british]{babel}
\usepackage[latin1]{inputenc}
\usepackage[T1]{fontenc}
\usepackage{newcent,helvet}
\renewcommand{\ttdefault}{cmtt}

\usepackage[it,footnotesize]{caption2}
\setlength{\abovecaptionskip}{5pt} % Space before caption
\setlength{\belowcaptionskip}{5pt} % Space after caption

\usepackage[bf,sf,small,nonindentfirst]{titlesec}
\newcommand{\sectionbreak}{\clearpage}
%\titleformat{\subsubsection}{\normalfontitshape}{\thesubsubsection}{.5em}{}
%\titlespacing{\subsubsection}{0pt}{*1}{*-1}
\usepackage{fancyhdr}
\usepackage{fancybox}

\usepackage{alltt}

\usepackage{graphicx}
\usepackage{parskip,a4,vmargin}
\setpapersize{A4}
\setmargrb{20mm}{15mm}{20mm}{15mm}

\frenchspacing
\sloppy

%%% Page headings
\pagestyle{fancy}
\renewcommand{\sectionmark}[1]{% Don't upcase the section title
  \markright{\thesection.\ #1}}
\fancyhead{}                    % clear header
\fancyhead[LE,RO]{\thepage}
\fancyhead[CE]{\textsc{The MHD code}}
\fancyhead[CO]{\rightmark}
%
\fancyfoot{}

% ---------------------------------------------------------------------- %

%%% Macros

%% Bold face \tt prompts (only works within `alltt' or \tt environment)
\newcommand{\prompt}[1]{{\ttfamily\bfseries{}#1}}

%% keys, names, paths, files, etc.
\newcommand{\code}[1]{\texttt{#1}}
\newcommand{\kbd}[1]{\texttt{\textsl{#1}\/}}
\newcommand{\key}[1]{{\setlength{\fboxsep}{1pt}\ovalbox{\sf #1}}}
\newcommand{\samp}[1]{`\code{#1}'}
\newcommand{\var}[1]{\textsl{#1}\/}
\newcommand{\env}[1]{\code{#1}}
\newcommand{\file}[1]{`\texttt{#1}'}
\newcommand{\command}[1]{\code{#1}}
\newcommand{\cmd}[1]{\command{#1}}
\newcommand{\option}[1]{`\command{#1}'}
\newcommand{\dfn}[1]{\textsl{#1}\/}
%\newcommand{\cite}[1]{}
\newcommand{\acronym}[1]{\textsc{#1}}
%\newcommand{\url}[1]{}
\newcommand{\email}[1]{\code{#1}}

\newcommand{\name}[1]{\textsl{#1}\/}
\newcommand{\Path}[1]{\file{#1}}

%
\newcommand{\bsT}{{\fontencoding{T1}\selectfont{\symbol{92}}}}
\newcommand{\bcks}{{\symbol{92}}}
\newcommand{\bs}{\bcks}       % Save us creation of a couple of fonts

%% Maths operators
% \newcommand{\arcosh} {\mathop{\rm arcosh}\nolimits}
% \newcommand{\arcoth} {\mathop{\rm arcoth}\nolimits}
% \newcommand{\sgn}    {\mathop{\rm sgn}\nolimits}
\newcommand{\grad}    {\mathop{\rm grad}\nolimits}
\newcommand{\Div}     {\mathop{\rm div}\nolimits}
\newcommand{\curl}    {\mathop{\rm curl}\nolimits}
\newcommand{\rot}     {\curl}
\newcommand{\Laplace} {\mathop{\Delta}\nolimits}
\newcommand{\erfc}    {\mathop{\rm erfc}\nolimits}
\newcommand{\erf}     {\mathop{\rm erf}\nolimits}

\newcommand{\vekt}[1] {\mathbf{#1}}
\newcommand{\const}   {\mbox{\rm const}}

%% Maths variables
\newcommand{\Av}            {\vekt{A}}
% \newcommand{\av}            {\vekt{a}}

\newcommand{\Bv}            {\vekt{B}}
% \newcommand{\bv}            {\vekt{b}}

\newcommand{\Cool}          {{\cal C}}
\newcommand{\cs}            {c_{\rm s}}
\newcommand{\csnull}        {c_{{\rm s},0}}

% \newcommand{\Ev}            {\vekt{E}}
% \newcommand{\ev}            {\vekt{e}}
% \newcommand{\ex}            {\ev_{x}}
% \newcommand{\ey}            {\ev_{y}}
% \newcommand{\ez}            {\ev_{z}}

% \newcommand{\Fv}            {\vekt{F}}

\newcommand{\gv}            {\vekt{g}}

\newcommand{\Heat}          {{\cal H}}

\newcommand{\jv}            {\vekt{j}}

% \newcommand{\nullvekt}      {\vekt{0}}

\newcommand{\Ra}            {\mathrm{Ra}}
\newcommand{\Reynolds}      {\mathrm{Re}}
\newcommand{\Rm}            {\mathrm{Rm}}

\newcommand{\uv}            {\vekt{u}}

% \newcommand{\Vol}           {{\cal V}}
% \newcommand{\vA}            {v_{\rm A}}

\newcommand{\xv}            {\vekt{x}}

% \newcommand{\zerovect}      {\nullvekt}

% ---------------------------------------------------------------------- %

\title{{\sffamily\bfseries Installing and Using the MHD code}}
%\subtitle{A very preliminary manual}
\author{Axel Brandenburg \& Wolfgang Dobler}

% ====================================================================== %

\begin{document}
\pagestyle{empty}

%\maketitle

\begin{titlepage}
  \begin{center}

  \large

  \vspace*{3cm}

  {\Large\sffamily\bfseries Installing and Using the MHD code}

  \vspace{0.5cm}

  {\sffamily A very preliminary manual}

  \vspace{1.5cm}

  {Axel Brandenburg \& Wolfgang Dobler}


  \vspace{2cm}

  \emph{\today}


\end{center}

\end{titlepage}


\newpage
\mbox{}
\vfill

Copyright \copyright{} 2001,2002 Axel Brandenburg \& Wolfgang Dobler 
\bigskip

Permission is granted to make and distribute verbatim copies of
this manual provided the copyright notice and this permission notice
are preserved on all copies.

Permission is granted to copy and distribute modified versions
of this manual under the conditions for verbatim copying,
provided that the entire resulting derived work is distributed under the
terms of a permission notice identical to this one.


\clearpage
\pagestyle{plain}
\pagenumbering{roman}

\tableofcontents
\clearpage
\pagestyle{fancy}
\pagenumbering{arabic}

\chapter{Obtaining the Code}

To obtain the MHD code, you should use \name{CVS} (concurrent version
system, see \url{http://fill.this.in}).
You proceed as follows:

\begin{enumerate}

\item Get a \name{CVS} client from \url{http://fill.this.in} and install it

\item Get the password for anonymous \name{CVS} access to the MHD code

\item Set your environment variable \env{CVSROOT} (you probably want to
  put this into your \file{.cshrc} or \file{.profile} setup files):
  \begin{alltt}
  \prompt{csh> } setenv CVSROOT \bs
        :pserver:anonymous@norserv.nordita.dk:/home/brandenb/CVS \
  \end{alltt}
  (all in one line, without the backslash) or
  \begin{alltt}
  \prompt{sh> } CVSROOT=:pserver:anonymous@norserv.nordita.dk:/home/brandenb/CVS
  \prompt{sh> } export CVSROOT \
  \end{alltt}

\item Log in:
  \begin{alltt}
  \prompt{unix> } cvs login
  cvs pasword: ........ \
\end{alltt}

\item Go to wherever you want the code:
  \begin{alltt}
  \prompt{unix> } cd \file{somewhere} \
  \end{alltt}

\item Check the code out:
  \begin{alltt}
  \prompt{unix> } cvs checkout -d mhd f90/pencil_modular \
  \end{alltt} 
  This creates a subdirectory \file{mhd} of your current directory
  \file{somewhere} and populates it with the MHD code's subdirectories.

\end{enumerate}


% ====================================================================== %

\section{Getting Started}

To get yourself started, you should run the comvection example which is
provided as the default configuration of the code.


% ====================================================================== %

\section{The Equations}

The equations solved by the MHD code are basically the standard
compressible MHD equations. However, the modular structure allows to
implement different versions of the MHD equations, as well as to switch
some of the equations off.

% ---------------------------------------------------------------------- %

\subsection{Continuity equation}

\begin{equation}
  \frac{D\ln\varrho}{Dt}
  = - \Div\uv \; .
\end{equation}

Here $\varrho$ denotes density, $\uv$ the fluid velocity, $t$ is time and
$D/Dt \equiv \partial/\partial t + \uv\cdot\grad$ is the convective
derivative.

% ---------------------------------------------------------------------- %

\subsection{Equation of motion}

\begin{equation}
  \frac{D\uv}{Dt}
   =  -\cs^2\grad\biggl(\frac{s}{c_p} + \ln\varrho\biggr)
      - \grad\Phi
      + \frac{\jv\times\Bv}{\varrho}
      + \nu \left( \Laplace\uv + \frac{1}{3}\grad\Div\uv \right) \; .
\end{equation}
Here, $\cs^2 = \gamma p/\varrho$ is the (squared) sound speed,
$\gamma=c_p/c_v$ the ratio of specific heats of \emph{adiabatic index},
$\Phi$ is the gravity potential, $\jv$ the electric current density, $\Bv$
the magnetic flux density, and $\nu$ is kinematic viscosity.
Note that the viscous term used here is only correct if the dynamical
viscosity $\mu \equiv \varrho\nu = \const$ everywhere.


% ---------------------------------------------------------------------- %
\subsection{Induction equation}

\begin{equation}
  \frac{\partial\Av}{\partial t}
  = \uv\times\Bv + \eta\mu_0\jv \; .
\end{equation}

Here $\Av$ is the magnetic vector potential, $\Bv = \curl\Av$ the magnetic
flux density, $\eta = 1/(\mu_0\sigma)$ is the magnetic diffusivity
($\sigma$ denoting the electrical conductivity), and $\mu_0$ the
magnetic vacuum permeability.


% ---------------------------------------------------------------------- %

\subsection{Entropy equation}

The current thermodynamics module \file{entropy} formulates the thermal
part of the physics in terms of \emph{entropy} $s$, rather than thermal
energy $e$, which you may be more familiar with.
Thus the two fundamental thermodynamical variables are $\ln\varrho$
and $s$.
The reason for this choice of variables is that entropy is the natural
physical variable for (at least) convection processes: the sign of the
entropy gradient determines convective (in)stability, the
\emph{Rayleigh number} is proportional to the entropy gradient, etc.

\begin{equation}
  \varrho T\frac{Ds}{Dt}
   =  \Heat - \Cool
      + \Div(\lambda\grad T)
      + \mu_0 \jv^2
      + 2\varrho\nu {\sf S}^2 \; .
\end{equation}

Here, $T$ is temperature, $c_p$ the specific heat at constant pressure,
$\Heat$ and $\Cool$ are explicit heating and cooling terms,
$\lambda$ is the thermal conductivity, and
\begin{equation}
  {\sf S}_{ik} = \frac{\partial_i u_k + \partial_k u_i}{2}
                 -\frac{1}{3} \delta_{ik}\Div\uv
\end{equation}
is the shear tensor.

    
% ====================================================================== %

\section{Using the Code}

Here's how to use the code \ldots



    
% ====================================================================== %

\section{Experimenting with the \LaTeX{} macros}


\bs{}code:    \code{call remove\_file()}

\bs{}kbd:     \kbd{M-x comment-region}

\bs{}key:     \key{F1}

\bs{}samp:    \samp{a}, \samp{e}, \samp{i}, \samp{o}, \samp{u}

\bs{}var:     \var{ivisc}

\bs{}env:     \env{CVSROOT}

\bs{}file:    \file{~/tmp/var.dat}

\bs{}command: \command{rm -f *}

\bs{}option:  \option{-l}, \option{--long-listing}

\bs{}dfn:     A \dfn{definition} is a specification sufficiently obfuscated
           to be miunderstood 

\bs{}cite:    See \cite{Abramowitz & Stegun}

\bs{}acronym: \acronym{MNRAS}

\bs{}url:     \url{http://www.nowhere.net/second_page.html}

\bs{}email:   \email{nobody@nowhere.nil}




% ====================================================================== %

\section{Index}



\end{document}

% End of file manual.tex
