%%%%%%%%%%%%%%%%%%%%%%            -*-LaTeX-*-
%%%   manual.tex   %%%
%%%%%%%%%%%%%%%%%%%%%%
%%
%%  Date:   06-Mar-2002
%%  Author: wd (Wolfgang.Dobler@kis.uni-freiburg.de)
%%  Description: 
% 
\ifx\pdfoutput\undefined        % not running pdflatex
  \documentclass[12pt,twoside,notitlepage,a4paper]{article}
\else                           % running pdflatex
  \documentclass[pdftex,12pt,twoside,notitlepage,a4paper]{article}
\fi

%\usepackage{url} %(do we not need this; but it's not working anyway)
\usepackage[bookmarks=false]{hyperref}
%\usepackage{german,a4}
%\usepackage[german,british]{babel}
\usepackage[latin1]{inputenc}
\usepackage[T1]{fontenc}
\usepackage{newcent,helvet}
\renewcommand{\ttdefault}{cmtt}

\usepackage{makeidx}

\usepackage[it,footnotesize]{caption2}
\setlength{\abovecaptionskip}{5pt} % Space before caption
\setlength{\belowcaptionskip}{5pt} % Space after caption

\usepackage[bf,sf,small,nonindentfirst]{titlesec}
\newcommand{\sectionbreak}{\clearpage}
%\titleformat{\subsubsection}{\normalfontitshape}{\thesubsubsection}{.5em}{}
%\titlespacing{\subsubsection}{0pt}{*1}{*-1}
\usepackage{fancyhdr}
\usepackage{fancybox}
\setcounter{tocdepth}{6} % Older versions of fancybox very annoyingly (and
                         % unnecessarily) resets this and make table of
                         % contents disappear

\usepackage{booktabs}
\usepackage{longtable}

\usepackage{alltt}

\usepackage{graphicx}
\usepackage{parskip,a4,vmargin}
%\setpapersize{A4}  %AB: what if somebody prints it in Santa Barbara?
%\setmargrb{20mm}{15mm}{20mm}{15mm}
\setmargrb{20mm}{25mm}{20mm}{15mm}  %AB: just trying to adjust it for here.

\frenchspacing
\sloppy

\makeindex


%%% Page headings
\pagestyle{fancy}
\renewcommand{\sectionmark}[1]{% Don't upcase the section title
  \markright{\thesection.\ #1}}
\fancyhead{}                    % clear header
\fancyhead[LE,RO]{\thepage}
\fancyhead[CE]{\textsc{The MHD code}}
\fancyhead[CO]{\rightmark}
%
\fancyfoot{}

% ---------------------------------------------------------------------- %

%%% Macros

%% Bold face \tt prompts (only works within `alltt' or \tt environment)
\newcommand{\prompt}[1]{{\ttfamily\bfseries{}#1}}

%% Margin and inline notes and remarks
\newcommand\note[1]{\marginpar{\renewcommand{\baselinestretch}{0.8}
        \raggedright\scriptsize\usefont{OT1}{phv}{mc}{n} #1}}
\newcommand{\Note}[1]{\emph{[#1]}}


%% keys, names, paths, files, etc.
\newcommand{\code}[1]{\texttt{#1}}
\newcommand{\kbd}[1]{\texttt{\textsl{#1}\/}}
\newcommand{\key}[1]{{\setlength{\fboxsep}{1pt}\ovalbox{\sf #1}}}
\newcommand{\samp}[1]{`\code{#1}'}
\newcommand{\var}[1]{\textsl{#1}\index{#1}\/}
\newcommand{\env}[1]{\code{#1}\index{#1}}
\newcommand{\file}[1]{`\texttt{#1}'}
\newcommand{\command}[1]{\code{#1}\index{#1}}
\newcommand{\cmd}[1]{\command{#1}}
\newcommand{\option}[1]{`\command{#1}'}
\newcommand{\dfn}[1]{\textsl{#1}\index{#1}\/}
%\newcommand{\cite}[1]{}
\newcommand{\acronym}[1]{\textsc{#1}\index{#1}}
%\newcommand{\url}[1]{}
\newcommand{\email}[1]{\code{#1}}

\newcommand{\name}[1]{\textsl{#1}\index{#1}\/}
\newcommand{\Path}[1]{\file{#1}}

%
\newcommand{\bsT}{{\fontencoding{T1}\selectfont{\symbol{92}}}}
\newcommand{\bcks}{{\symbol{92}}}
\newcommand{\bs}{\bcks}       % Save us creation of a couple of fonts

%% Maths operators
% \newcommand{\arcosh} {\mathop{\rm arcosh}\nolimits}
% \newcommand{\arcoth} {\mathop{\rm arcoth}\nolimits}
% \newcommand{\sgn}    {\mathop{\rm sgn}\nolimits}
\newcommand{\grad}    {\mathop{\rm grad}\nolimits}
\newcommand{\Div}     {\mathop{\rm div}\nolimits}
\newcommand{\curl}    {\mathop{\rm curl}\nolimits}
\newcommand{\rot}     {\curl}
\newcommand{\Laplace} {\mathop{\Delta}\nolimits}
\newcommand{\erfc}    {\mathop{\rm erfc}\nolimits}
\newcommand{\erf}     {\mathop{\rm erf}\nolimits}

\newcommand{\vekt}[1] {\mathbf{#1}}
\newcommand{\const}   {\mbox{\rm const}}

%% Maths variables
\newcommand{\Av}            {\vekt{A}}
% \newcommand{\av}            {\vekt{a}}

\newcommand{\Bv}            {\vekt{B}}
% \newcommand{\bv}            {\vekt{b}}

\newcommand{\Cool}          {{\cal C}}
\newcommand{\cs}            {c_{\rm s}}
\newcommand{\csnull}        {c_{{\rm s},0}}

% \newcommand{\Ev}            {\vekt{E}}
% \newcommand{\ev}            {\vekt{e}}
% \newcommand{\ex}            {\ev_{x}}
% \newcommand{\ey}            {\ev_{y}}
% \newcommand{\ez}            {\ev_{z}}

% \newcommand{\Fv}            {\vekt{F}}

\newcommand{\gv}            {\vekt{g}}

\newcommand{\Heat}          {{\cal H}}

\newcommand{\jv}            {\vekt{j}}

% \newcommand{\nullvekt}      {\vekt{0}}

\newcommand{\Ra}            {\mathrm{Ra}}
\newcommand{\Reynolds}      {\mathrm{Re}}
\newcommand{\Rm}            {\mathrm{Rm}}

\newcommand{\uv}            {\vekt{u}}

% \newcommand{\Vol}           {{\cal V}}
\newcommand{\vA}            {v_{\rm A}}

\newcommand{\xv}            {\vekt{x}}

% \newcommand{\zerovect}      {\nullvekt}

% ---------------------------------------------------------------------- %

\title{{\sffamily\bfseries Installing and Using the High-Order MHD MPI code}}
%\subtitle{A very preliminary manual}
\author{Wolfgang Dobler \& Axel Brandenburg}
\date{\today,~ $ $Revision: 1.7 $ $}

% ====================================================================== %

\begin{document}
\pagestyle{empty}

%\maketitle

\begin{titlepage}
  \begin{center}

  \large

  \vspace*{3cm}

  {\Large\sffamily\bfseries Installing and Using the High-Order MHD MPI code}

  \vspace{0.5cm}

  {\sffamily A very preliminary manual}

  \vspace{1.5cm}

  {Wolfgang Dobler \& Axel Brandenburg}


  \vspace{2cm}

  \emph{\today,~ $ $Revision: 1.7 $ $}


\end{center}

\end{titlepage}


\newpage
\mbox{}
\vfill

Copyright \copyright{} 2001,2002 Wolfgang Dobler \& Axel Brandenburg
\bigskip

Permission is granted to make and distribute verbatim copies of
this manual provided the copyright notice and this permission notice
are preserved on all copies.

Permission is granted to copy and distribute modified versions
of this manual under the conditions for verbatim copying,
provided that the entire resulting derived work is distributed under the
terms of a permission notice identical to this one.


\clearpage
\pagestyle{plain}
\pagenumbering{roman}

\section*{Foreword}

This code was first developed at the Turbulence Summer School of the
Helmholtz Institute in Potsdam (2001).
Often people ask us whether our code is publicly available;
that's why we decided to make it a bit more user friendly (than it was before)
and to put it on a public cvs repository.

The code is primarily designed to deal with weakly compressible turbulent
flows: that's why we use high-order first and second derivatives.
In order for the code to be easily parallelizable we use explicit
(as opposed to compact) finite differences.
Typical scientific targets include driven MHD turbulence in a periodic box,
convection in a slab with non-periodic upper and lower boundaries,
a convective star embedded in a fully nonperiodic box, accretion disc turbulence in
the sheering sheet approximation, etc.

We use the magnetic vector potential to ensure that the field
is always divergence-free. At the same time, having the magnetic
vector potential readily available is a tremendous advantage if
one wants to monitor the magnetic helicity, for example.
This code is therefore particularly well suited for all kind of
dynamo problems.

The code is non-conservative. Thus, conserved quantities should only be
conserved up to the discretization error of the scheme (not the machine acuracy). Fully conservative
schemes cannot easily deal with large temperature and density contrasts
and there is no guarantee that a conservative code is more accurate with
respect to quantities that are not explicitly conserved, such as entropy.
Magnetic helicity is another important example of a quantity that is to
our knowledge not guaranteed to be conserved by ordinary flux conserving
schemes.

We don't have any immediate plans to implement adaptive mesh refinement
into the code. The main reason is that this would cause a major technical
complication. Moreover, turbulence is generically space-filling, so any
local refinement to smaller scales may not be useful. On the other hand,
turbulence may well be confined to certain regions in space, so one
could indeed gain by solving the outer regions with fewer points.
So we'll keep that in mind.

In order to be cache-efficient and to allow the free use of temporary
variables without using up lots of memory, we solve the equations along
pencils in the $x$ direction. Ghost zones are used to implement boundary
conditions on physical and processor boundaries. The domain can be tiled
evenly in the $y$ and $z$ directions. MPI calls are used to communicate
between processors. For testing purposes a dummy version can be invoked
to run on just one processor (e.g.\ on a laptop).

The various physics is invoked by linking the corresponding modules in the
Makefile. The main reason for this procedure is to have a convenient way
of stripping off extra ballast from the code, so it compiles faster and
uses less memory. It is also a convenient way of changing the physics
without using messy switches.

The parameter input file (\var{run.in}) is read whenever there exists the file
\var{RELOAD} in the run directory. This allows one
\footnote{OK, some of this should be moved to section 3.}
to change parameters during the fly. This can be useful if one
wants to stop the in a graceful way (ie reduce number of timesteps),
if one wants to change the length of the time step,
the viscosity or other diffusion parameters,
or (in forced calculations) the sound speed. Obviously, in
some cases changing certain parameters on the fly could be totally crazy.

We will be happy to include user-supplied changes and updates to the code
in future releases, so please get in touch with us.

\vspace{5mm}
%\noindent
\var{Wolfgang.Dobler@kis.uni-freiburg.de}\\
\var{Axel.Brandenburg@nordita.dk}

\tableofcontents
\clearpage
\pagestyle{fancy}
\pagenumbering{arabic}


% ====================================================================== %

\section{Obtaining the Code}

To obtain the MHD code, you should use \name{CVS} (concurrent version
system, see \url{http://fill.this.in}).
You proceed as follows:

\begin{enumerate}

\item Get a \name{CVS} client from \url{http://fill.this.in} and install it

\item Get the password for anonymous \name{CVS} access to the MHD code

\item Set your environment variable \env{CVSROOT} (you probably want to
  put this into your \file{.cshrc} or \file{.profile} setup files):
  \begin{alltt}
  \prompt{csh> } setenv CVSROOT \bs
        :pserver:anonymous@norserv.nordita.dk:/home/brandenb/CVS \
  \end{alltt}
  (all in one line, without the backslash) or
  \begin{alltt}
  \prompt{sh> } CVSROOT=:pserver:anonymous@norserv.nordita.dk:/home/brandenb/CVS
  \prompt{sh> } export CVSROOT \
  \end{alltt}

\item Log in:
  \begin{alltt}
  \prompt{unix> } cvs login
  cvs password: ........ \
\end{alltt}

\item Go to wherever you want the code:
  \begin{alltt}
  \prompt{unix> } cd \file{somewhere} \
  \end{alltt}

\item Check the code out:
  \begin{alltt}
  \prompt{unix> } cvs checkout -d mhd f90/pencil_modular \
  \end{alltt} 
  This creates a subdirectory \file{mhd} of your current directory
  \file{somewhere} and populates it with the MHD code's subdirectories.

\end{enumerate}


% ====================================================================== %

\section{Getting Started}

To get yourself started, you should run the convection example which is
provided as the default configuration of the code.

It is absolutely mandatory monitor exactly what one is doing.
IDL is a very useful tool to inspect the data in a very flexible fashion.

% ====================================================================== %

\section{Code structure}

% ---------------------------------------------------------------------- %

\subsection{Directory tree}


% ---------------------------------------------------------------------- %

\subsection{Basic concepts}


\subsubsection{Data access in pencils}


\subsubsection{Modules}


\subsubsection{Localization (adapt-mkfile)}



% ====================================================================== %

\section{The Equations}

The equations solved by the MHD code are basically the standard
compressible MHD equations. However, the modular structure allows to
implement different versions of the MHD equations, as well as to switch
some of the equations off.

% ---------------------------------------------------------------------- %

\subsection{Continuity equation}

\begin{equation}
  \frac{D\ln\varrho}{Dt}
  = - \Div\uv \; .
\end{equation}

Here $\varrho$ denotes density, $\uv$ the fluid velocity, $t$ is time and
$D/Dt \equiv \partial/\partial t + \uv\cdot\grad$ is the convective
derivative.

% ---------------------------------------------------------------------- %

\subsection{Equation of motion}

\begin{equation}
  \frac{D\uv}{Dt}
   =  -\cs^2\grad\biggl(\frac{s}{c_p} + \ln\varrho\biggr)
      - \grad\Phi
      + \frac{\jv\times\Bv}{\varrho}
      + \nu \left( \Laplace\uv + \frac{1}{3}\grad\Div\uv \right) \; .
\end{equation}
Here, $\cs^2 = \gamma p/\varrho$ is the (squared) sound speed,
$\gamma=c_p/c_v$ the ratio of specific heats of \emph{adiabatic index},
$\Phi$ is the gravity potential, $\jv$ the electric current density, $\Bv$
the magnetic flux density, and $\nu$ is kinematic viscosity.
Note that the viscous term used here is only correct if the dynamical
viscosity $\mu \equiv \varrho\nu = \const$ everywhere.


% ---------------------------------------------------------------------- %
\subsection{Induction equation}

\begin{equation}
  \frac{\partial\Av}{\partial t}
  = \uv\times\Bv - \eta\mu_0\jv \; .
\end{equation}

Here $\Av$ is the magnetic vector potential, $\Bv = \curl\Av$ the magnetic
flux density, $\eta = 1/(\mu_0\sigma)$ is the magnetic diffusivity
($\sigma$ denoting the electrical conductivity), and $\mu_0$ the
magnetic vacuum permeability.


% ---------------------------------------------------------------------- %

\subsection{Entropy equation}

The current thermodynamics module \file{entropy} formulates the thermal
part of the physics in terms of \emph{entropy} $s$, rather than thermal
energy $e$, which you may be more familiar with.
Thus the two fundamental thermodynamical variables are $\ln\varrho$
and $s$.
The reason for this choice of variables is that entropy is the natural
physical variable for (at least) convection processes: the sign of the
entropy gradient determines convective (in)stability, the
\emph{Rayleigh number} is proportional to the entropy gradient, etc.

\begin{equation}
  \varrho T\frac{Ds}{Dt}
   =  \Heat - \Cool
      + \Div(\lambda\grad T)
      + \eta\mu_0 \jv^2
      + 2\varrho\nu {\sf S}^2 \; .
\end{equation}

Here, $T$ is temperature, $c_p$ the specific heat at constant pressure,
$\Heat$ and $\Cool$ are explicit heating and cooling terms,
$\lambda$ is the thermal conductivity, and
\begin{equation}
  {\sf S}_{ik} = \frac{\partial_i u_k + \partial_k u_i}{2}
                 -\frac{1}{3} \delta_{ik}\Div\uv
\end{equation}
is the shear tensor.

    
% ====================================================================== %

\section{Boundary conditions}

    
% ====================================================================== %

\section{Using the Code}
\label{Input-params}

Here's how to use the code \ldots

% ---------------------------------------------------------------------- %
\subsection{Start parameters}
Here is a list of parameters that are set in \file{start.in}

Any variable referred to as a \dfn{flag} can be set to any nonzero value
to switch the corresponding feature on.

Not all parameters are used for a given scenario.

% ---------------------------------------------------------------------- %
\begin{longtable}{lp{0.6\textwidth}}
%\begin{tabular}{lp{0.6\textwidth}}
\toprule
  \multicolumn{1}{c}{\emph{Variable}}
               & \multicolumn{1}{c}{\emph{Meaning}} \\
\midrule
  \var{ip}     & (anti-)verbosity level: \code{ip=1} produces lots of
                 diagnostic output, \code{ip=14} virtually none. \\
  \var{[x-z]0},
  \var{L[x-z]},
  \var{iper[x-z]},
               & determine the geometry of the box.
                 If \code{iperx=0} (box \emph{not\/} periodic in $x$), $x$
                 ranges from $x_0$ to $x_0+L_x$; if \code{iperx=1}
                 (periodic box in $x$), the rightmost point is
                 $x_0+L_x-\delta x$. In all cases, three ghost zones will
                 be added. \\
  \var{z1}, \var{z2}, \var{ztop}
               & specific to the solar convection case.
                 The stable layer is $z_0 < z < z_1$, the unstable layer
                 $z_1 < z < z_2$, and the top (isothermal) layer is
                 $z_2 < z < z_{\rm top}$
                 \Note{How is this related to $L_z$??} \\
  \var{hcond[0-2]}, \var{whcond}
               & specific to the solar convection case: heat conductivities
                 $\lambda$ in the individual layers. \var{hcond0} is the
                 value $\lambda_{\rm unst}$ in the unstable layer,
                 \var{hcond1} is the ratio
                 $\lambda_{\rm stab}/\lambda_{\rm unst}$ for the stable
                 layer, and \var{hcond2} is the ratio 
                 $\lambda_{\rm top}/\lambda_{\rm unst}$ for the top layer.
                 The function $\lambda(z)$ is not discontinuous, as the
                 transition between the different values is smoothed over
                 the width \var{whcond}. \\
  \var{mpoly[0-2]}, \var{isothtop}
               & specific to the solar convection case: polytropic indices
                 of unstable (\var{mpoly0}), stable (\var{mpoly1}) and top
                 layer (\var{mpoly2}).
                 If the flag \var{isothtop} is set, the
                 top layer is initialized to be isothermal, otherwise
                 thermal (plus hydrostatic) equilibrium is assumed for all
                 three layers, which results in a piecewise polytropic
                 stratification. \\
  \var{ampl}   & ??? \\
  \var{init}   & ??? \\
  \var{urand}  & amplitude of initial random velocity fluctuations \\
  \var{gamma}  & adiabatic index of the gas (typically $\gamma=5/3$ in
                 astrophysical applications \\
  \var{cs0}, \var{rho0}
               & reference values of sound speed $\cs$ and density
                 $\varrho$. In the convection case, these are the values of
                 $\cs$ and $\varrho$ at $z=z_{\rm top}$. \\
  \var{gravz}  & vertical gravity $g_z$; normally $<0$. \\
  \var{grads0} & [Not used for convection reference case] initial entropy
                 gradient. \\
\bottomrule
%\end{tabular}
\end{longtable}
% ---------------------------------------------------------------------- %

% ---------------------------------------------------------------------- %
\subsection{Run parameters}
\label{Run-params}

Here is a list of parameters that are set in \file{run.in}.

Note that formatted \name{Fortran} input allows for parameters to be
skipped (indicated by possible whitespace, followed by a comma), which
causes the corresponding variable to retain the value it had before the
\cmd{read} command was called.
In most cases, this default value will be the one set in \file{start.in}.

Any variable referred to as a \dfn{flag} can be set to any nonzero value
to switch the corresponding feature on.

Not all parameters are used for a given scenario.


% ---------------------------------------------------------------------- %
\begin{longtable}{lp{0.6\textwidth}}
\toprule
  \multicolumn{1}{c}{\emph{Variable}}
               & \multicolumn{1}{c}{\emph{Meaning}} \\
\midrule
  \var{nt}     & number of time steps to carry out. \\
  \var{it1}    & write monitoring output every \var{it1} time steps.\\
  \var{dt}     & time step $\delta t$ if positive,
                 $\delta t/\delta t_{\rm Courant}$ if negative. A typical
                 value is \code{dt=-0.4}. Using less can be useful as a
                 temporary fix until the source of the problem is found.\\
  \var{isave}  & update snapshot file \file{var.dat} every \var{isave}
                 time steps. \\
  \var{iorder} & order of time step. Set \code{iorder=1} for Euler
                 stepping, \code{iorder=3} for the 3rd-order Runge-Kutta
                 (2$N$) scheme. \code{iorder=1} is useful for debugging,
                 because one wants to know what happens after the first
                 or second step. \code{iorder=2} is also possible, but
                 not \code{iorder>3}.\\
  \var{dsnap}  & save full snapshots \file{VAR$N$} every \var{dsnap} time
                 units. \\
  \var{dvid}   & write two-dimensional sections for generation of videos
                 every \var{dsnap} time units. \\
  \var{dforce} & force every \var{dforce} time units [???]. \\
  \var{dtmin}  & abort if time step $\delta t < \delta t_{\rm min}$. \\
  \var{tinit}  & ??? \\
  \var{tdamp}  & damp velocities in the initial time interval
                 $0 < t < t_{\rm damp}$. \\
  \var{dampu}  & strength of initial damping of velocity. \\
  \var{dampuext}, \var{rdamp}, \var{wdamp}
               & permanently damp velocities at $|\xv| > r_{\rm damp}$
                 with strength \var{dampuext}, with a smooth transition
                 between damped and undamped of radial width \var{wdamp}. \\
  \var{ip}     & (anti-)verbosity level: \code{ip=1} produces lots of
                 diagnostic output, \code{ip=14} virtually none. \\
  \var{i[x-z]} & monitor values of some variables in point with indices
                 $(i_x,i_y,i_z)$. \\
  \var{cs}, \var{rho}
               & reference values of sound speed $\cs$ and density
                 $\varrho$. In the convection case, these are the values of
                 $\cs$ and $\varrho$ at $z=z_{\rm top}$. \\
  \var{ivisc}  & switch for viscosity term
                 \Note{which value gives which term?} \\
  \var{cdtv}   & weight of diffusive/viscous relative to advective time
                 step for determining the Courant step
                 \begin{equation}
                   \delta t_{\rm Cou}
                   = \min\left( \frac{\delta x_{\rm min}}
                                     {U_{\rm max}} ,
                                c_{\delta t_v}
                                \frac{\delta x_{\rm min}^2}
                                     {D_{\rm max}}
                         \right) \; ,
                 \end{equation}
                 where
                 $\delta x_{\rm min} \equiv \min(\delta x, \delta y, \delta z)$;
                 $U_{\rm max} \equiv \max\left(|\uv|
                                     + \sqrt{\cs^2{+}\vA^2}\right)$,
                 $\cs$ and $\vA$ denoting sound speed and Alfv\'en speed,
                 respectively;
                 and $D_{\rm max} = \max(\nu,\chi,\eta)$, where
                 $\nu$ denotes kinematic viscosity,
                 $\chi = \lambda/(c_p\varrho)$ thermal diffusivity and
                 $\eta$ the magnetic diffusivity.
                 \\
  \var{hcond[0-2]}, \var{whcond}
               & heat conductivities, see Sec~\ref{Input-params}; leave
                 these empty unless you want to override the settings from
                 \file{start.in}. \\
  \var{cdiffrho}
               & ??? \\
  \var{gravz}  & vertical gravity, see Sec~\ref{Input-params}; leave
                 this empty unless you want to override the setting from
                 \file{start.in}. \\
  \var{cheat}, \var{wheat}
               & heating term: add total amount \var{cheat} of heat per
                 time unit in zone near bottom of width \var{wheat}.
                 You probably rather want to use \var{Fheat} (see below). \\
  \var{cool}, \var{wcool}
               & cooling term: cool top layer of width \var{wcool} to
                 reference temperature (given by ${\cs}_0$) by cooling term
                 of strength \var{cool}. \\
  \var{Fheat}  & heat flux through bottom boundary. \\
  \var{iforce}, \var{force}, \var{relhel}
               & on-off flag, strength and kinetic helicity of forcing
                 terms. \\
  \var{bc[x-z]}
               & boundary conditions. Set these to a sequence of letters 
                 like `p,p,p,p,p,p,p,p' for periodic boundaries, or
                 `s,s,a,a2,c1:c2,s,s,a' for non-periodic ones.
                 Each entry (between commas) corresponds to one of the
                 variables, normally these are $u_x$, $u_y$, $u_z$,
                 $\ln\varrho$, $s/c_p$, $A_x$, $A_x$ and $A_x$, in this
                 order.
                 \begin{description}
                 \item[\option{p}] indicates periodicity
                 \item[\option{a}] indicates antisymmetry w.\,r.\,t.~the
                   boundary, i.\,e.~vanishing value
                 \item[\option{s}] indicates symmetry w.\,r.\,t.~the
                   boundary, i.\,e.~vanishing first derivative
                 \item[\option{a2}] indicates antisymmetry w.\,r.\,t.~the
                   arbitrary value on the boundary, i.\,e.~vanishing
                   second derivative
                 \item[\option{c1}] is a special boundary condition for
                   $\ln\varrho$ and $s$, which ensures constant heat flux
                   through the boundary
                 \item[\option{c2}] is a special boundary condition for
                   $\ln\varrho$ and $s$, which ensures constant
                   temperature at the boundary 
                 \end{description}
                 The special syntax $a$:$b$ (e.\,g.~`\code{c1:c2}') means: use
                 boundary condition $a$ at the left/lower boundary, but
                 $b$ at the right/upper one. 
                 \\
  \var{form1}  & format for monitoring output. typically something like
                 \code{'(i8,f11.5,1p,50(e10.3," "))'} for printing number
                 of time step, time, and then a lot of different
                 quantities. Make sure you allow for sufficiently many
                 quantities here (you can always use a large multiplier to
                 be on the safe side). \\
\bottomrule
\end{longtable}
% ---------------------------------------------------------------------- %

\section{FAQs}

\begin{itemize}
\item {\it What's so cool about the fort.20 file?}
\item 
\end{itemize}

\section{Timings}

Below we quote the wall clock time per mesh point (including the ghost zones)
and per full 3-stage time step.

Linux 500 MHz: 22 $\mu$s/pt/step in reference implementation (21-Mar-02/AB)
%CVS:  run.f90               version 1.18          of 2002/03/05 17:43:13 
%CVS:  register.f90          version 1.15          of 2002/03/08 15:43:19 
%CVS:  entropy.f90           version 1.33          of 2002/03/09 20:18:55 
%CVS:  magnetic.f90          version 1.6           of 2002/01/30 16:56:30 
%CVS:  grav_z.f90            version 1.5           of 2002/01/23 19:56:13 

Memory usage: 7.3 MB for $32\times32\times64$ MHD simulation.
%7288 with 32x32x64 (single processor, linux)
%print,38*38*71.*8.*2.*4./1e6

% ====================================================================== %

\section{Experimenting with the \LaTeX{} macros}


\bs{}code:    \code{call remove\_file()}

\bs{}kbd:     \kbd{M-x comment-region}

\bs{}key:     \key{F1}

\bs{}samp:    \samp{a}, \samp{e}, \samp{i}, \samp{o}, \samp{u}

\bs{}var:     \var{ivisc}

\bs{}env:     \env{CVSROOT}

\bs{}file:    \file{~/tmp/var.dat}

\bs{}command: \command{rm -f *}

\bs{}option:  \option{-l}, \option{--long-listing}

\bs{}dfn:     A \dfn{definition} is a specification sufficiently obfuscated
           to be misunderstood 

\bs{}cite:    See \cite{Abramowitz & Stegun}

\bs{}acronym: \acronym{MNRAS}

\bs{}url:     \url{http://www.nowhere.net/second_page.html}

\bs{}email:   \email{nobody@nowhere.nil}




% ====================================================================== %

\printindex

\vfill\bigskip\noindent{\tiny\it
$ $Id: manual.tex,v 1.7 2002-03-21 17:38:14 brandenb Exp $ $}


\end{document}

% End of file manual.tex
