%% $Id$
%% This file was automatically generated by Pencil::DocExtractor,
%% so think twice before you modify it.
%%
%% Source files:
%%   boundcond.f90

% ---------------------------------------------------------------- %
\begin{longtable}{lp{0.8\textwidth}}
\toprule
  \multicolumn{1}{c}{\emph{Variable}} & {\emph{Meaning}} \\
\midrule
  \multicolumn{2}{c}{Module \file{boundcond.f90}} \\
\midrule
  \var{p}         & periodic \\
  \var{pp}        & periodic across the pole \\
  \var{ap}        & anti-periodic across the pole \\
  \var{s}         & symmetry symmetry, $f_{N+i}=f_{N-i}$; \\
  \var{ss}        & symmetry, plus function value given \\
  \var{s0d}       & symmetry, function value such that df/dy=0 \\
  \var{a}         & antisymmetry \\
  \var{a2}        & antisymmetry relative to boundary value \\
  \var{v}         & vanishing third derivative \\
  \var{1s}        & onesided \\
  \var{cT}        & constant temp. \\
  \var{sT}        & symmetric temp. \\
  \var{asT}       & select entropy for uniform ghost temperature
                    matching fluctuating boundary value,
                    $T_{N-i}=T_{N}=$;
                    implies $T'(x_N)=T'(x_0)=0$ \\
  \var{f}         & freeze value \\
  \var{s+f}       & freeze value \\
  \var{fg}        & ``freeze'' value, i.e. maintain initial \\
  \var{1}         & f=1 (for debugging) \\
  \var{set}       & set boundary value \\
  \var{e1}        & extrapolation \\
  \var{e2}        & extrapolation \\
  \var{der}       & set derivative on the boundary \\
  \var{cop}       & outflow: copy value of last physical point to
                    all ghost cells \\
  \var{c+k}       & no-inflow: copy value of last physical point
                    to all ghost cells, but suppressing any inflow \\
  \var{sfr}       & stress-free boundary condition for spherical coordinate system. \\
  \var{nfr}       & Normal-field bc for spherical coordinate system.
                    Some people call this the ``(angry) hedgehog bc''. \\
  \var{pfc}       & perfect conducting boundary condition along $\theta$ boundary \\
%
\bottomrule
\end{longtable}

