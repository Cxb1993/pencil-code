%%%%%%%%%%%%%%%%%%%%%%%%%            -*-LaTeX-*-
%%%   multigrid.tex   %%%
%%%%%%%%%%%%%%%%%%%%%%%%%
%%
%% Date:   21-Jun-2007
%% Author: wd (Wolfgang.Dobler@ucalgary.ca)
%% Description: 
%%   

%% Choose appropriate driver:
\RequirePackage{ifpdf}
\ifpdf
  \def\mydriver{pdftex}         % writing PDF
\else
  \def\mydriver{dvips}          % writing DVI
\fi

\documentclass[\mydriver,12pt,twoside,notitlepage]{article}

%\usepackage{german,a4}
%\usepackage[german,british]{babel}
%\usepackage[latin1]{inputenc}
%\usepackage[T1]{fontenc}\usepackage{lmodern}
\usepackage[scaled]{helvet}
\renewcommand{\familydefault}{phv}

\usepackage{amsmath,amssymb,bm,wasysym}
%\usepackage{latexsym,exscale}%\usepackage{newcent,helvet}%\usepackage{pxfonts}%\usepackage{txfonts}
%\usepackage[it,footnotesize]{caption2}
%\setlength{\abovecaptionskip}{5pt} % Space before caption
%\setlength{\belowcaptionskip}{5pt} % Space after caption

%% Load titlesec before hyperref or \section links will be off with dvipdfm
%\usepackage[bf,sf,small,nonindentfirst]{titlesec}
%\newcommand{\sectionbreak}{\clearpage}
%\titleformat{\subsubsection}{\normalfont\itshape}{\thesubsubsection}{.5em}{}
%\titlespacing{\subsubsection}{0pt}{*1}{*-1}

\usepackage{units}

\usepackage{graphicx,color}
\usepackage{multicol}
\usepackage{parskip}
\usepackage[paperwidth=20cm,paperheight=15cm,hmargin=20mm,top=15mm,bottom=15mm,twosideshift=5mm]{geometry}

\usepackage{fancyvrb}

\usepackage{makeidx}
\usepackage[bookmarksopen=true,bookmarksopenlevel=2]{hyperref} % should come late/last

\ifpdf
  \usepackage{pdfpages}
\else
  \newcommand{\includepdf}[2][]{%
%     \fbox{%
%       \vfill\\%
%       \centerline{#2}%
%       \vfill\\%
%       }%
    \centerline{%
      \fbox{%
        \parbox[c][0.5\textheight]{0.5\textwidth}{%
          \vspace*{\stretch{1}}%
          \hfill Include file \path{#2}\hfill\mbox{}%
          \vspace*{\stretch{2}}%
        }%
      }%
    }%
  }%
  \fi

\definecolor{DarkBlue}{rgb}{0,0,0.7}

%% Fancyvrb customizations
\DefineShortVerb{\|}
\newcommand{\BoxLabel}[1]{\fbox{\rmfamily\emph{#1}}}
%% A customized verbatim environment
\DefineVerbatimEnvironment%
  {CodeVerbatim}%
  {Verbatim}%
  {frame=single,%
   framesep=10pt,%
   xleftmargin=4ex,%
   xrightmargin=4ex,%   
   commandchars=\\\{\},%
   gobble=2%
  }


\frenchspacing
\sloppy
%%% Multiple, three-column indexes
\makeindex
%% The following is adapted from hyperref.sty and fixes hyperrefs in the
%% index after all of our nasty manipulations:
% \makeatletter
%   \@ifpackageloaded{hyperref}{%
%     \let\HyInd@org@wrindex\@wrindex
%     \def\@wrindex#1#2{\HyInd@@wrindex{#1}#2||\\}%
%     \def\HyInd@@wrindex#1#2|#3|#4\\{%
%       \ifx\\#3\\%
%         \HyInd@org@wrindex{#1}{#2|hyperpage}%
%       \else
%         \def\Hy@temp@A{#3}%
%         \ifx\Hy@temp@A\HyInd@ParenLeft
%           HyInd@org@wrindex{#1}{#2|#3hyperpage}%
%         \else
%           \HyInd@org@wrindex{#1}{#2|#3}%
%         \fi
%       \fi
%     }%
%   }{}
% \makeatother
%% Redefine index to be in three columns (adapted from `index.sty'):
\makeatletter
\renewenvironment{theindex}{%
  %\edef\indexname{\the\@nameuse{idxtitle@\@indextype}}%
  \if@twocolumn\@restonecolfalse
  \else\@restonecoltrue
  \fi
  \columnseprule \z@
  \columnsep 35\p@
  \begin{multicols}{3}[\section*{\indexname}%
    %\ifx\index@prologue\@empty%
    %\else\index@prologue\bigskip
    %\fi
  ]%
  \@mkboth{\MakeUppercase\indexname}%
          {\MakeUppercase\indexname}%
  \thispagestyle{plain}%
  \parindent\z@
  \parskip\z@ \@plus .3\p@\relax
  \let\item\@idxitem
}
{ \end{multicols}
  \if@restonecol\onecolumn\else\clearpage\fi
}
\makeatother

%% Matrices and tensors
\newcommand{\matx}[1]{\mbox{\boldmath${\cal #1}$\unboldmath}}
\newcommand{\tensor}[1]{\matx{#1}}

%% Math macros
\newcommand{\Av}      {\mathbf{A}}

\newcommand{\Bv}      {\mathbf{B}}


\newcommand{\etav}    {\bm{\eta}}
\newcommand{\Ev}      {\mathbf{E}}

\newcommand{\fv}      {\mathbf{f}}

\newcommand{\gv}      {\mathbf{g}}

\newcommand{\jv}      {\mathbf{j}}

\newcommand{\kv}      {\mathbf{k}}

\newcommand{\Laplace} { \mathop{\Delta}\nolimits}

\newcommand{\mtildei} {\widetilde{m}_{\rm i}}

\newcommand{\uv}      {\mathbf{u}}
\newcommand{\uve}     {\uv_{\rm e}}
\newcommand{\uvi}     {\uv_{\rm i}}

\newcommand{\vv}      {\mathbf{v}}
\newcommand{\vve}     {\vv_{\rm e}}

\newcommand{\xv}{\mathbf{x}}

% ---------------------------------------------------------------------- %

\title{Long Time Steps for the \textsc{Pencil Code}\\[1.5ex]
  {\large Implementing and testing Runge--Kutta--Chebyshev schemes\\
    for highly diffusive problems}
}
\author{Wolfgang Dobler}

% ---------------------------------------------------------------------- %

\begin{document}
\thispagestyle{empty}

\maketitle

\clearpage
\tableofcontents
\clearpage

%%%%%%%%
\section{Introduction: Runge--Kutta Schemes}
%%%%%%%%

%%%
\subsection{Classical explicit schemes}
Explicit Runge--Kutta scheme:
\begin{align}
  \tau_1 &= t_0 \; ,
            & \etav_1 &= \yv_0 \; ,
                         & \kv_1 &= h\fv(\tau_1, \etav_1) \; ,\\
  \tau_2 &= t_0 + c_2 h \; ,
            & \etav_2 &= \yv_0 + a_{21} \kv_1 \; ,
                         & \kv_2 &= h\fv(\tau_2, \etav_2) \; ,\\
  \tau_3 &= t_0 + c_3 h \; ,
            & \etav_3 &= \yv_0 + a_{31} \kv_1 + a_{32} \kv_2 \; ,
                         & \kv_3 &= h\fv(\tau_3, \etav_3) \; ,\\
  & \vdots &  & \vdots  &  &\vdots \nonumber \\
  \tau_s &= t_0 + c_s h \; ,
            & \etav_s &= \yv_0 + a_{s1} \kv_1 + a_{s2} \kv_2
             + \ldots + a_{s,s-1} \kv_{s-1} \; ,
                              & \kv_s &= h\fv(\tau_s, \etav_s) \; ,\\
  t_1 &= t_0 + h \; ,
            & \yv_1   &= \yv_0 + b_1 \kv_1 + b_2 \kv_2
                         + \ldots + b_{s-1} \kv_{s-1} + b_{s} \kv_{s} \; .
\end{align}


Butcher tableau:
\begin{equation}
  \begin{array}{c|ccccc}
    0                                                 \\
    c_2    & a_{21}                                   \\
    c_3    & a_{31} & a_{32}                          \\
    \vdots & \vdots &        & \ddots &               \\
    c_s    & a_{s1} & a_{s2} & \cdots & a_{s,s-1}     \\
    \hline
    (1)    & b_1    & b_2    & \cdots & b_{s-1} & b_s
  \end{array}
\end{equation}

where all omitted elements $a_{ij}$ vanish.\footnote{
  For embedded schemes there will be two lines with
  coefficients $b_i$ and $\tilde{b}_i$:
  \begin{equation}
    \begin{array}{c|ccccc}
      0                                                 \\
      c_2    & a_{21}                                   \\
      c_3    & a_{31} & a_{32}                          \\
      \vdots & \vdots &        & \ddots &               \\
      c_s    & a_{s1} & a_{s2} & \cdots & a_{s,s-1}     \\
      \hline
             & b_1    & b_2    & \cdots & b_{s-1} & b_s \\
      \hline
             & \tilde{b}_1
                      & \tilde{b}_2
                               & \cdots
                                        & \tilde{b}_{s-1}
                                                  & \tilde{b}_s
    \end{array}
  \end{equation}
  The sets of $b$ correspond to two different schemes of different order,
  which are combined in order to estimate the error.
}


An implicit $s$-stage Runge--Kutta (of order $p$) for the system of
ordinary differential equations
\begin{equation}
  \dot{\yv} = \fv(t,\yv)
\end{equation}
is given by
\begin{align}
  \tau_i  &= t_0 + c_j h \; ,
             & \etav_i &= \yv_0 + \sum\limits_{j=1}^{s} a_{ij} \kv_j \; ,
                          & \kv_j &= h\fv(\tau_i, \etav_j) \; ,\\
  t_1     &= t_0 + h \; ,
             & \yv_1   &= \yv_0 + \sum\limits_{j=1}^{s} b_j \kv_j \; .
\end{align}

All schemes typically use (at least) first-order accurat sub-steps:
\begin{equation}
  \sum\limits_{j=1}^{s} a_{ij} = c_i \; , \qquad i=1,2,\ldots,s \; .
\end{equation}

The coefficients $c_j$, $a_{ij}$, $b_j$ are conveniently represented in a
\emph{Butcher tableau},
\begin{equation}
  \begin{array}{c|c}
    \cv & \Av \\
    \hline
        & \bv^{\mathsf T}
  \end{array}
  \quad=\quad
  \begin{array}{c|ccccc}
    c_1    & a_{11} & a_{12} & a_{13} & \cdots & a_{1s} \\
    c_2    & a_{21} & a_{22} & a_{23} & \cdots & a_{2s} \\
    c_3    & a_{31} & a_{32} & a_{33} & \cdots & a_{3s} \\
    \vdots & \vdots &        &        & \ddots &        \\
    c_s    & a_{s1} & a_{s2} & a_{s3} & \cdots & a_{ss} \\
    \hline
    (1)    & b_1    & b_2    & b_3    & \cdots & b_s
  \end{array}
\end{equation}

Implicit Runge--Kutta schemes require the solution of nonlinear systems of
equations in each substep, but have desirable stability properties.

Explicit schemes are far easier to implement.



%%%
\subsection{$2N$-schemes}

%%%
\subsection{Stability}

\begin{itemize}
  \item Stability polynomial
  \item Plots
\end{itemize}

% ---------------------------------------------------------------------- %

%%%%%%%%
\section{Runge--Kutta--Chebyshev schemes}
%%%%%%%%

We can do better \ldots{}
Sounds familiar? $\rightarrow$ Chebyshev

$\rightarrow$ RKC-1 stability polynomial
-- but that's no scheme yet.


% ---------------------------------------------------------------------- %

%%%%%%%%
\section{Example}
%%%%%%%%

Kinematic $\alpha^2$-dynamo:

$\alpha=\mathrm{const.}$ in a sphere $r<1$,
\qquad
$\eta = \begin{cases}
    1 & \text{for $r<1$} \; , \\
    8 & \text{for $r>1$}\\
  \end{cases}$

\medskip

{\small\color{DarkBlue}
\begin{CodeVerbatim}[label=\BoxLabel{run.in}]
  &run_pars
    ! For eta_ext=8, stability limits are
    !   itorder=3:  dt_eta=1.69e-5
    !   RKC-40:     dt_eta=7.03e-3
    dt=3.5e-3, tmax=2.0
  /
  &magnetic_run_pars
    iresistivity='shell', wresistivity=0.03,
    eta=1.0, eta_ext=8.0
    ! For eta=1.0, eta_ext=4.0, critical alpha_effect is around 11.3
    lmeanfield_theory=T, alpha_effect=15, alpha_quenching=0.
    alpha_profile='const', lweyl_gauge=T
  /
\end{CodeVerbatim}
}

[Resolution + geometry\ldots]

% ---------------------------------------------------------------------- %

%%%
\subsection{Comparison}
\begin{enumerate}
\item Standard 3rd-order Runge--Kutta scheme (Williamson)\\
  \texttt{dt=7.00e-6}
\item RKC-20\\
  \texttt{dt=4.375e-4}
\item RKC-40\\
  \texttt{dt=3.50e-3}
\end{enumerate}
(each at $\approx 1/2\,\delta t_{\rm max}$).

Speed:
\begin{enumerate}
\item $\unit[71.9]{\mu s}$ per step + grid pt   
  (at 98.7\%)
  \RedArrow $\unit[0.021]{\mu s}$ per step + time unit
\item $\unit[35.0]{\mu s}$ per step + grid pt
  (at 98.7\%)
  \RedArrow $\unit[0.080]{\mu s}$ per step + time unit
\item $\unit[4.77]{\mu s}$ per step + grid pt
  (at 98.2\%)
  \RedArrow $\unit[0.68]{\mu s}$ per step + time unit
\end{enumerate}
\bigskip

{\small\color{DarkBlue}
\begin{CodeVerbatim}[label=\BoxLabel{top}]
   PID USER      PR  NI  VIRT  RES  SHR S %CPU %MEM    TIME+  COMMAND           
  3693 wdobler   20   0  818m 196m 1916 R   99 10.4  21:30.87 run.x              
  3741 wdobler   20   0  697m  96m 1900 R   95  5.1  18:22.56 run.x              
\end{CodeVerbatim}
}


[Growth rate changed from 53.62 (RK-40) to 53.74 (RK-20)]

\end{document}

% End of file multigrid.tex

%%% Please leave this for Emacs [wd]:

%% Local Variables:
%% ispell-check-comments: t
%% Local IspellDict: canadian
%% End:

% LocalWords:  ifpdf pdftex dvips
