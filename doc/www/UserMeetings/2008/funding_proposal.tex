\documentclass{article}

\usepackage{url}
\usepackage{a4wide}
\usepackage{color}
\usepackage{eurosym}
\usepackage{hyperref}

\begin{document}

\section*{Pencil Code Meeting 2008 in Leiden}

\subsection*{Abstract}

The Pencil Code Meeting is an annual meeting dedicated to the Pencil
Code. The purpose of the meeting is to bring regular users and core
developers together to discuss scientific and technical progress since
the last meeting, to instigate collaborative projects and to allow new
users to learn more about the code and to interact with other users
and developers

\subsection*{Scientific summary}

The Pencil
Code\footnote{\url{http://www.nordita.org/software/pencil-code/}} is a
highly versatile and modular code for massively parallel computations. The
project is unique in the astrophysical community in that the code is
continuously developed by a number of independent researchers. The development
is kept in alignment through the use of a version control system (CVS) that
allows the developers to work on different aspects of the code, while still
working within a single branch. Automated nightly checks of a large number of
physical standard problems ensures that no new development will break the
code.
\\ \\
The Pencil Code is primarily designed to deal with weakly compressible
turbulent flows, which is why we use high-order spatial derivatives.  To
achieve good parallelization, we use explicit (as opposed to compact) finite
differences. Typical scientific targets include driven MHD turbulence in a
periodic box, convection in a slab with non-periodic upper and lower
boundaries, a convective star embedded in a fully non-periodic box, accretion
disc turbulence in the shearing sheet approximation, planetesimal formation by
self-gravity in turbulent flows, etc. The code is parallelised using the MPI
(Message Passing Interface) library and has been successfully tested on up to
1,000 processors at $1024^3$ grid points, with near linear scaling.
\\ \\
The Pencil Code can be downloaded freely from the web-page$^1$, supplying the
astrophysical community with an ideal tool for solving the partial differential
equations of hydrodynamics and magnetohydrodynamics. The Pencil Code community
consists of approximately 10-15 main developers and 500 regular users who have
downloaded the code from the web page. Registred users can update their
copy of the code with the latest version through CVS.
\\ \\
The Pencil Code Meeting 2008 will take place August 19-22 2008 at the
Sterrewacht Leiden in The Netherlands. The annual Pencil Code Meeting is a workshop
that brings the core developers and regular users together to discuss recent
progress and future challenges related to the Pencil Code. Pencil
Code meetings have taken place since 2005 and have proved crucial for keeping
developers aligned about future priorities and for introducing new users to the
code.
\\ \\
Here are some topics that the meeting aims at covering:
\begin{itemize}
  \item Recent science results obtained with the Pencil Code
  \item A Pencil Code paper
  \item Technical issues, such as transition from CVS to SVN version
    control systems (scheduled for May)
  \item Future improvements to the code
  \item Exchange of experience and ideas
\end{itemize}
The website of the meeting can be found at\\
\url{http://www.strw.leidenuniv.nl/~ajohan/pencil2008/}.

\subsection*{Provisional list of participants}

Currently 13 people have registred online or otherwise expressed that they are
going to come:
\begin{center}
  \begin{tabular}{lll}
    Last name & First name & Institution \\
    \hline
    Babkovskaia &  Natalia &  University of Oulu\\
    Bingert &  Sven &  Kiepenheuer Institute for Solar Physics\\
    Brandenburg &  Axel &  Nordita\\
    Dintrans &  Boris &  Observatoire Midi-Pyr\'en\'ees\\
    Dobler &  Wolfgang & Private company \\
    Heinemann & Tobias & DAMPT, Cambridge \\
    Johansen &  Anders &  Sterrewacht Leiden\\
    K\"apyl\"a &  Petri &  Helsinki University Observatory\\
    Klahr &  Hubert &  Max-Planck-Institut f\"ur Astronomie\\
    Lyra &  Wladimir &  Uppsala University\\
    Oishi & Jeff & UC Berkeley \\
    Remacle &  Vincent &  Universit\'e Libre de Bruxelles\\
    Sundberg &  Mikaela &  Stockholm University\\
  \end{tabular}
\end{center}
Since there are still more than three months until the workshop, we expect the
number of participants to rise to at least 15 (with 1-2 participants from
outside of Europe). Last year's meeting in
Stockholm\footnote{See
\url{http://agenda.albanova.se/conferenceDisplay.py?confId=185}} had 19
participants.

\subsection*{Meeting programme}

The Pencil Code Meeting 2008 will be focused on discussion sessions and
exchange of ideas in an informal workshop environment. Talks will primarily
be scheduled for morning sessions, while the afternoons are dedicated to
predefined discussion topics.
\\ \\
The preliminary programme is:\\
\begin{tabular}{|c|c|c|c|c|}
  \hline
  & Tuesday 19/08 & Wednesday 20/08 & Thursday 21/08  & Friday 22/08 \\
  \hline
          &              & Talk (Dobler)  & Talk (Dintrans)    & Talk (Klahr) \\
  Morning & Registration & Talk (Bingert) & Talk (Babkovskaia) & Talk (P\"apyl\"a) \\
          &              & Talk (Lyra)    & Talk (Heinemann)   & Talk
          (Johansen)\\
  \hline
            & Welcome (Johansen) & Discussion: & Discussion: & Discussion: \\
            & Talk (Brandenburg) & ``Changing from CVS & ``Postprocessing &
            Miscellaneous \\
  Afternoon & Discussion: & to SVN'' & with Python'' & topics \\
            & ``A Pencil Code paper'' & ``Improving the & & \\
            & ``An experience'' &  Makefile''                 & & \\
            & database'' & & & \\
  \hline
\end{tabular}

\subsection*{Budget}

An estimated 10 of the 15 participants will be eligible for travel
support, either because they hold junior positions (PhD students, young
postdocs) or because they hold positions in institutions that are not able to
provide travel support.
\\ \\
The estimated costs of the workshop are (in Euro):\\
\begin{center}
  \begin{tabular}{lcrr}
    Travel       &:& 3000 & (10 $\times$ 300) \\
    Accommodation &:& 3200 & (10 $\times$ 4 $\times$ 80) \\
    Lunch        &:&   600 & (15 $\times$ 4 $\times$ 10) \\
    \hline
    Total        &:& 7200
  \end{tabular}
\end{center}
Lunch would be provided for every participant, independently of whether
they receive travel support.
\\ \\
Sterrewacht Leiden has kindly offered to provide a conference room for the
entire duration of the workshop, free of charge.

\subsection*{Co-funding}

Funding requests will be submitted to the following organisations:
\begin{itemize}
  \item Astrosim (\euro{3800})\\
    \url{http://www.astrosim.net/}
  \item NOVA (\euro{1500})\\
    \url{http://www.strw.leidenuniv.nl/nova/}
  \item Leids Kerkhoven-Bosscha Fonds (\euro{1500})\\
    \url{http://www.strw.leidenuniv.nl/lkbf/index.php?node=84}
\end{itemize}

\subsection*{Curriculum vitae of scientific organiser}

\begin{tabular}{ll}
  \textbf{Name}: & Anders Johansen \\
  \textbf{Current position}: & Postdoctoral fellow at Leiden University\\
  \textbf{Birth data}: & 18 February 1977 in Copenhagen \\
  \textbf{Nationality}: & Danish \\
  \textbf{Address}: & Coebelweg 5, 2324 KX Leiden, The Netherlands\\
  \textbf{E-mail}: & \url{ajohan@strw.leidenuniv.nl} \\
  \textbf{Telephone number}: &
   (+31) 71 527 8454 (office) / (+31) 648 211 427 (mobile) \\
  \textbf{Website}: &
  \url{http://www.strw.leidenuniv.nl/~ajohan}
\end{tabular}
\vspace{0.3cm}
\\ \\
{\it Work experience}:
\begin{itemize}
  \item 02.2008-\textcolor{white}{01.2010} : Postdoctoral fellow at Leiden
    University, Leiden, The Netherlands
  \item 07.2007-01.2008 : Postdoc at Max Planck Institute for Astronomy,
  Heidelberg, Germany
\end{itemize}
{\it Education}:
\begin{itemize}
  \item 2004-2007 PhD (astrophysics) at Max Planck Institute for Astronomy\\
    PhD thesis: ``Numerical models of the early stages of planet formation''\\
    Supervisors: Thomas Henning and Hubert Klahr\\
    Graduated with summa cum laude
  \item 2002-2004 Master's degree (astronomy) at Copenhagen University/NORDITA\\
    Master's thesis: ``Ice condensation, dust coagulation and vortex activity in
    protoplanetary discs''\\
    Supervisors: \AA ke Nordlund, Axel Brandenburg and Anja C.  Andersen
  \item 1998-2002 Bachelor's degree (astronomy/physics) at Copenhagen
    University\\
    Bachelor thesis: ``Detections of planetary transits with the GAIA
    satellite'' (GAIA-CUO-106)\\
    Supervisor: Erik H\o g
  \item 1994-1996 High school at \O regaard Gymnasium, Hellerup, Copenhagen\\
\end{itemize}
\newpage
\noindent{\it Research interests}:
\begin{itemize}
  \item Protoplanetary accretion discs; magnetic and non-magnetic
    turbulence; dust dynamics and diffusion; self-gravity and particle-mesh
    methods; code development; supercomputing
\end{itemize}
\noindent{\it Awards and honours}:
\begin{itemize}
  \item Otto Hahn Medal (2008)\\
    Awarded by the Max Planck Society for outstanding PhD work (prize: one year
    fellowship outside of Germany)
  \item The Patzer Prize (2007)\\
    For best refereed paper by young scientist at MPIA in 2007 (\euro{2000})
  \item PhD graduation with ``summa cum laude'' grade in July 2007
  \item The Annette Kade Student Fellowship Program (2006)\\
    For 6 weeks scientific visit to the American Museum of Natural History
  \item The Patzer Prize (2005)\\
      For best refereed paper by young scientist at MPIA in 2005 (\euro{1000})
  \item Christian og Ottilia Brorsons Rejselegat (2004)\\
    For moving costs (DKK 10,000 $\approx$ \euro{1,500}).
\end{itemize}

\noindent {\it Short-term scientific stays:}
\begin{itemize}
  \item CITA, University of Toronto (2007)\\
    Visited Andrew Youdin (1 week)
  \item American Museum of Natural History (2006)\\
    Visited Mordecai-Mark Mac Low (6 weeks)
  \item Princeton University (2006)\\
    Visited Andrew Youdin (3 days)
\end{itemize}

\noindent{\it Organiser of workshops:}
\begin{itemize} 
  \item ``Pencil Code Meeting 2008'' (Leiden, August 2008, main organiser)
  \item ``MHD Days 2006'' (Heidelberg, December 2006, coorganiser)
  \item ``Ringberg workshop on planet formation'' (Ringberg, December 2004,
  coorganiser)
\end{itemize}

\noindent{\it Five most relevant scientific publications:}
\begin{itemize}
  \item ``Rapid planetesimal formation in turbulent circumstellar discs''\\
    {\bf Johansen A.}, Oishi J., Mac Low M.-M., Klahr H., Henning Th., \&
    Youdin A.\\
    Nature, vol.\ 448, p.\ 1022-1025
  \item ``Protoplanetary disc turbulence driven by the streaming instability:
    Non-linear saturation and particle concentration''\\
    {\bf Johansen A.}, \& Youdin A.\\
    The Astrophysical Journal, vol.\ 662, p.\ 627-641
  \item ``Dust sedimentation and self-sustained Kelvin-Helmholtz turbulence in
    protoplanetary disc mid-planes'' (2006)\\
    {\bf Johansen A.}, Henning Th., \& Klahr H.\\
    The Astrophysical Journal, vol.\ 643, p.\ 1219-1232
  \item ``Dust diffusion in protoplanetary discs by magnetorotational
    turbulence'' (2005)\\
    {\bf Johansen A.}, \& Klahr H.\\
    The Astrophysical Journal, vol.\ 634, p.\ 1353-1371
  \item ``Simulations of dust-trapping vortices in protoplanetary discs''
    (2004)\\
    {\bf Johansen A.}, Andersen A.\ C., \& Brandenburg A.\\
    Astronomy and Astrophysics, v.\ 417, p.\ 361-374\\
    Cover image of A\&A vol.\ 417 / 1 - April I - 2004
\end{itemize}

\end{document}


\end{document}
