%%%%%% Document general properties %%%%%%

\documentclass[a4paper,12pt]{article}
\usepackage{fixltx2e}[2006/03/24]  % package fixing some LaTeX bugs
\usepackage[american]{babel}
\usepackage[a4paper, dvips=true, scale=0.86, nofoot, centering]{geometry}
\usepackage[utf8]{inputenc}
%\usepackage[T1]{fontenc}
\usepackage{microtype}
\usepackage{color}

%%%%%% Sciences addons  packages %%%%%
%\usepackage{achemso}
\usepackage{amsmath}
\usepackage{amssymb}
%\usepackage{amsbsy}
%\usepackage{chemarr}
%\usepackage{chemarrow}




%%%%%%%%  Graphic packages  %%%%%%%%%%%%
\usepackage[dvips]{graphicx}   % final mode
%\usepackage[draft,pdftex]{graphicx}   % draft mode
%\usepackage[dvips]{color}
%\usepackage{psfrag} 



%%%%%%%%%%% Fonts packages %%%%%%%%%%%%%%
%\usepackage{euler}
%\usepackage{mathrsfs}
%\usepackage{times}
%\renewcommand{\ttdefault}{txtt}
%\usepackage{frmath}
%\usepackage{marvosym}


%%%%%%%%%%%  float/tables packages %%%%%%%%%%%%%
%\usepackage{multirow}
%\usepackage{multicol}
%\usepackage{arydshln}
%\usepackage{float}
%\usepackage{nonfloat}
%\usepackage{placeins}
%\usepackage[ruled,rm,sc,bf]{caption}
%\usepackage{afterpage}


%%%%%%%%%% Additional Packages %%%%%%%%%%
\usepackage{hyperref}

%\newcommand{\upd}{\genfrac{}{}{0pt}{0}{\nearrow}{\searrow}}
%\newcommand{\ffrac}[2]{\genfrac{}{}{0pt}{0}{#1}{#2}}


%%%%%%%%%%%%%%  Presentation properties %%%%%%%%%%

\pagestyle{headings}
%\sloppy
%\fussy
%\renewcommand{\baselinestretch}{2}
\setlength{\columnsep}{5mm}
\setcounter{tocdepth}{5}
\usepackage{indentfirst}

%\renewcommand{\thefootnote}{\alph{footnote}}
%\makeatletter
%\renewcommand{\@makefnmark}{\textsuperscript{\texttt \thefootnote}}
%\makeatother


\renewcommand{\descriptionlabel}[1]{$\longrightarrow$ '\texttt{#1}' :}
\newcommand{\note}[1]{\textcolor{red}{#1}}

%\floatstyle{ruled}
%\newfloat{scheme}{ht}{lsch}
%\floatname{scheme}{\textsc{Scheme}}
%\restylefloat{figure}

%%%%%%%%%%%%%% Bibliography  %%%%%%%%%%%%%%%%%%%%%

%\usepackage{footbib}
%\footbibliography{biblio}
%\footbibliographystyle{unsrt}
%\usepackage{overcite}


%%%%%%%%%%%%%%%%%%%%%%% Document %%%%%%%%%%%%%%%%%%%%%%%%%%%

\title{Description of the pencil-code HDF5 files \texttt{v0.1}}

\author{}


\hypersetup{%
  pdftitle = {Description of the pencil-code HDF5 files},
  pdfauthor = {Rapha\"el Plasson} ,
  pdfcreator = {\LaTeX\ with package \flqq hyperref\frqq} 
}


\begin{document}


\maketitle
\thispagestyle{empty}

%\tableofcontents

%\listoffigures

\section{Data structure}
\label{sec:data-structure}

In the following description, the full pathway for both subgroups and
datasets is given for better readability. The subgroups are written
with a trailing '$\slash$' to be distinguished from the datasets. The
attributes relative to groups and datasets are written without the
pathway. The different elements are presented in this order: after
each subgroup follow its different attributes, then its datasets,
then its embedded subgroups.


Please note that this is a first working version (thus the
\texttt{v0.1}...), intended to be a proposal to be discussed. It
should be considered as experimental, and thus shouldn't be expected
to be stable (i.e.\  its format may evolve a lot).


\textbf{Important :} Please update this file immediately when you
update the pencil-code HDF5 format, so that this description is always
correctly describing the current implementation.

\note{Notes and questions to be discussed are written in red. I
  suggest to increase the version number of the HDF5 file to
  \texttt{1.0} once we will have agreed and get rid of all these
  notes.}

\begin{figure}[p] \centering
  % \fbox{
  \parbox{18cm}{\scriptsize %Yes, it's small, the purpose is to try to
                            %have the whole structure on one single page...
    \begin{description}
    \item[/]  Root of the file
      \begin{description}
      \item[name] String Attribute. Set to ``PencilCode''. This attribute
        is intended to describe the type of data stocked there,
        i.e.\  pencil code data. Any informations concerning the run itself
    are intended to be stocked in the respective subgroups.
  \item[ver] String Attribute. Version of the
    pencil-code HDF5 file. The format described in this file
    corresponds to the version \texttt{v0.1}. 
  \item[dateC] String Attribute. Date of creation of the
    file. All dates are formatted as a string typeset as
      'DD/MM/YY'. \note{Maybe there are specific type format to stock dates,
      but I couldn't find it.}  
  \item[dateM] String Attribute. Date of last modification of the
    file.
  \item[/param/] Subgroup intended to stock all the parameters data
    relative to the runs. \note{I assume that all the relevant
      parameters can be found in \texttt{params.log} in datafile. The interest
      of reading the parameters there is that we have not only all the
      parameters obtained during init, but also before \emph{each} run,
      so that all parameters changes made after each run can be
      found.}
    \begin{description}
    \item[/param/init/] Subgroup containing the parameters
      passed during the \texttt{init} phase.
      \begin{description}
      \item[/param/init/init\_pars/]
        Subgroup of the parameters of
        \texttt{init\_pars}
        \begin{description}
          \item[/param/init/init\_pars/cvsid] String dataset of
            dimension (1,), non-resizable
          \item[/param/init/init\_pars/ip] Integer dataset of
            dimension (1,), non-resizable
          \item[/param/init/init\_pars/xyz0] Float dataset of
            dimension (3,), non-resizable
          \item[/param/init/init\_pars/...] All the parameters will be
            added. \note{It has been chosen to list each parameter in
            separate datasets rather than in a single large dataset
            array, because it gather different types (and sometimes
            even arrays of data)... but maybe this should be discussed,
            and a single array would be better ?}
        \end{description}
      \item[/param/init/hydro\_init\_pars/] Subgroup of the parameters
        of \texttt{hydro\_init\_pars}
        \begin{description}
        \item[/param/init/hydro\_init\_pars/inituu] String dataset of
            dimension (1,), non-resizable
        \item[/param/init/hydro\_init\_pars/...] etc.
        \end{description}
      \item[/param/init/...\_init\_pars/] etc. \note{Do you think that we
        should create the subgroups in all cases, or only in the cases
        there are referred to in the pencil code files (i.e. when they
        are really needed for the run). In the present implementation,
        I chose to only add the relevant subgroups, rather than add
        empty one.}
      \end{description}
    \item[/param/run/] Subgroup containing the parameters
      passed during each \texttt{run} phase. in all the arrays, the
      first dimension corresponds to the number of the run.
      \begin{description}
      \item[/param/run/timerun] Float dataset of
        dimension (\texttt{nbrun},1), first dimension resizeable, \texttt{nbrun} being
        the number of run that were realized. Note that this
        parameter do note appear as a pencil code parameter, but is
        written in the log file each time a new run is launched as a
        continuation from preceding runs.
      \item[/param/run/run\_pars/] Subgroup of the parameters of
        \texttt{run\_pars} 
        \begin{description}
        \item[/param/run/run\_pars/nt] Integer dataset of
            dimension (\texttt{nbrun},1), first dimension resizable
        \item[/param/init/run\_pars/...] etc.
        \end{description}
      \item[/param/run/chemistry\_run\_pars/] Subgroup of the
        parameters of \texttt{chemistry\_run\_pars} 
        \begin{description}
        \item[/param/run/chemistry\_run\_pars/mobility] Float dataset of
            dimension (\texttt{nbrun},\texttt{nbchem}), first
            dimension resizable. \texttt{nbchem} is the number of
            chemical species.
        \item[/param/init/chemistry\_run\_pars/...] etc.
      \item[/param/run/...\_run\_pars/] etc.
        \end{description}
      \end{description}
    \end{description}
  \item[/data/] Subgroup intended to stock all the data resulting from
    the runs.
    \begin{description}
    \item[/data/time\_series] Float dataset of the time series, of
      dimension (timeslices, nbvar),  \texttt{timeslices} corresponds
      to an output every \texttt{/param/run/run\_pars/it1} time
      steps, and \texttt{nbvar} corresponds to the number of
      diagnostic variables entered in the file \texttt{print.in}
    \item[/data/slices] Float dataset of the slices, of dimension
      (timeslices2,x,y), \texttt{timeslices2} corresponds to an output
      every \texttt{/param/run/run\_pars/dvid} unit of time, and
      \texttt{x,y} are the grid dimension in the chosen slice.
    \item[/data/var] Float dataset of the snapshots, of dimension
      (timeslices3,x,y,z), \texttt{timeslices3} corresponds to an
      output every  \texttt{/param/run/run\_pars/dsnap} unit of time,
      and \texttt{x,y,z} is the dimension of the whole grid
    \end{description}
  \item[/notes/] Subgroup intended to stock some additional
    informations about the runs.
    \begin{description}
    \item[/notes/note] Dataset of Variable Length Strings. Dimension:
      resizable (1,). This dataset is intended to stock written notes
      about the run. Each note can be of arbitrarily long size, and a
      arbitrary number of notes can be added. More precisely, each
      note is in HDF5 VL string format (VL standing for Variable
      Length), and the dataset is a resizable array of VL strings, of
      dimension one, initially created with one (empty) string
      element.
    \item[...] This subgroup should be left open to any personal
      modifications, so that any additional datasets or subgroups can
      be entered there as wanted.
    \end{description}
  \end{description}
\end{description}
}%}
\caption{Data structure of a pencil-code HDF5 file}
\label{fig:data-struct}
\end{figure}
\section{Example of use}
\label{sec:example-use}

All the following example are given in python, using the module
\texttt{h5py}. This module was chose rather than the other
implementation (\texttt{python-tables}) for being simple of use
through the high-level component, while at the same time offering
access  to the almost entire HDF5 C API through the low-level
components. There is nothing python-specific, so that the created
HDF5 files will (should...) be readable by any other HDF5
implementation, independently of the chosen language. Note that for
the moment, direct access to the HDF5 must be done (using high-level
compounds of h5py module), but I intend to write a python interface to
access and modify these data. 

All the following example are supposed to be written on the python
command line, after having imported the HDF5 module by:
\begin{verbatim}
>>> import h5py
>>>
\end{verbatim}
The ``\texttt{>>>}'' corresponds to the python prompt, and thus indicate
the commands typed by the user. The lines without the prompt are the
answers returned by python.


\subsection{Accessing a file}
\label{sec:opening-file}

A file may be open by the command \texttt{h5py.File}. Different
informations can be obtained from the returned object.
\begin{verbatim}
>>> f=h5py.File("datafile.hdf5","a")
>>> f
<HDF5 file "datafile.hdf5" (mode a, 3 root members)>
>>> f.attrs
<Attributes of HDF5 object "/" (4)>
>>> f.keys()
['data', 'notes', 'param']
>>>
\end{verbatim}

Be careful, all the modifications that are made to a file are
buffered. It is necessary to either flush or close the file if one
wants to be sure that everything has been written on the disk.
\begin{verbatim}
>>> f.flush()
>>> f.close()
>>>
\end{verbatim}

\subsection{Reading the file attribute}
\label{sec:read-file-attr}

Attributes of the file can be directly read, so that we can check that
this is a pencil code data file, which version of file format it is,
and when it was created and last modified:
\begin{verbatim}
>>> f.attrs["name"]
'PencilCode'
>>>> f.attrs["ver"]
'v0.1'
>>> f.attrs["dateC"]
'28/09/09'
>>> f.attrs["dateM"]
'28/09/09'
>>>
\end{verbatim}
Note that one can easily modify these attributes, or to create to
one. It is advisable to not do so !

\subsection{notes}
\label{sec:notes}

Initially only \texttt{f['notes/note'][0]} exists, and it is
set to the empty string \texttt{''}:
\begin{verbatim}
>>> f['/notes/note'][0]
''
>>> f['/notes/note'][1]
Traceback (most recent call last):
  File "<stdin>", line 1, in <module>
  (... snip here verbose error messages ...)
ValueError: Index (1) out of range (0-0)
>>>
\end{verbatim}

Any data can be entered in the note, and additional notes can be created:
\begin{verbatim}
>>> f['notes/note'][0]="This is a test file"
>>> f['notes/note'].resize((2,))
>>> f['notes/note'][1]="This is an additional note"
>>> f['notes/note'][1]
'This is an additional note'
>>> f['notes/note'][...]
array([This is a test file, This is an additional note], dtype=object)
>>> 
\end{verbatim}

Object can be used to directly access to the notes:
\begin{verbatim}
>>> note=f['notes/note']
>>> note[0]
'This is a test file'
>>> note
<HDF5 dataset "note": shape (2,), type "|O4">
>>> 
\end{verbatim}

\section{Filters}
\label{sec:filters}

Discussing the possible filters. Should be either possibility of
manual tuning, or fastest (null=no compression, data in order), fast
(shuffle+lzf=compression with optimization of access time) or small
(shuffle+gzip=best compression available). See benchmark on
\url{http://h5py.alfven.org/lzf/}. 

\end{document}
